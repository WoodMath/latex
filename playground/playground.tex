\documentclass[a4paper,10pt]{article}
%\documentclass[a4paper,10pt]{scrartcl}

\usepackage[utf8]{inputenc}

\title{CSE 3314 Notes}
\author{JeffGWood@mavs.uta.edu}
\date{Summer 2014}

\pdfinfo{%
  /Title    ()
  /Author   ()
  /Creator  ()
  /Producer ()
  /Subject  ()
  /Keywords ()
}

\begin{document}

\textbf{Equivalence Relateion}\newline



 A special type of binary relation, called an \textbf{\textit{equivalence relation}}, captures the notion of two objects being equal in some feature. A binary relation R is an equivalence relation if R satisfies three conditions: 
\newline
\newline\textbf{1.} \textit{R} is \textbf{\textit{reflexive}} if for every \textit{x}, \textit{xRx};
\newline\textbf{2.} \textit{R} is \textbf{\textit{symmetric}} if for every \textit{x} and \textit{y}, \textit{xRy} implies \textit{yRx}; and
\newline\textbf{3.} \textit{R} is \textbf{\textit{transitive}} if for every \textit{x}, \textit{y}, and \textit{z}, \textit{xRy} and \textit{yRz} implies \textit{xRz}.





%%\{V,E\} = \{\{1,2,3\},\{(1,2),(2,3),(3,1)\}\}

\end{document}
