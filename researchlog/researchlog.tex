%	Syntax from following sources
%		https://www.sharelatex.com/learn/Sections_and_chapters
%		https://www.sharelatex.com/learn/Table_of_contents
%		http://www.bibtex.org/Using/
%		https://www.economics.utoronto.ca/osborne/latex/BIBTEX.HTM
%		https://www.latex-tutorial.com/tutorials/beginners/latex-bibtex/
%		http://tex.stackexchange.com/questions/205/what-graphics-packages-are-there-for-creating-graphics-in-latex-documents
%		
% Main stuff
\documentclass[a4paper,10pt]{article}
\usepackage[utf8]{inputenc}

%% Math Packages
\usepackage{amssymb}
\usepackage{amsmath}
\usepackage{amsfonts}

%% Date Time Pakcages
\usepackage[USenglish]{babel}
\usepackage[nodayofweek,level]{datetime}
\usepackage[margin=0.5in]{geometry}

%% Formatting Packages
\usepackage{indentfirst}

%% Citation Packages
\usepackage{cite}
\usepackage{hyperref}

%% Table packages
\usepackage{longtable}

%% Commands section
\newcommand{\logentry}[4]{ \selectlanguage{USenglish} \formatdate{#2}{#1}{#3}  & {#4}  \\ \hline}


\title{Research Log}
\author{JeffGWood@mavs.uta.edu}
\date{\today}

\pdfinfo{
  /Title    (Research Log)
  /Author   (Jeff Wood)
  /Creator  ()
  /Producer ()
  /Subject  ()
  /Keywords ()
}

\begin{document}

	\maketitle
	\begin{longtable}{l p{12cm}}
		\hline
		\logentry{3}{30}{2016}{Established research log after 3 hours of learning new \LaTeX}
		\logentry{4}{2}{2016}{Added some additional comments to the \textbf{Process}}
		\logentry{4}{3}{2016}{Have been reading ~\cite{IBR-Book}. 
			\textbf{Have Question for Kamangar} regarding ~\cite{IBR-Book} about difference between:
			\begin{itemize}
				\item \textbf{Camera Plane} : Cooridinates \textit{u},\textit{v}
				\item \textbf{Focal Plane} : Cooridinates \textit{s},\textit{t}
			\end{itemize}
		}
		\logentry{4}{11}{2016}{Reviewing blog articles located at:
			\begin{itemize}
				\item \url{https://erget.wordpress.com/2014/02/01/calibrating-a-stereo-camera-with-opencv/}
				\item \url{https://erget.wordpress.com/2014/02/28/calibrating-a-stereo-pair-with-python/}
				\item \url{https://erget.wordpress.com/2014/03/13/building-an-interactive-gui-with-opencv/}
				\item \url{https://erget.wordpress.com/2014/04/27/producing-3d-point-clouds-with-a-stereo-camera-in-opencv/}
			\end{itemize}
			for process to get webcam up and running. Previous issues related to fine-tuning \textit{block matching} parameters. Need to review sources at list at bottom of \url{http://docs.opencv.org/2.4/modules/calib3d/doc/camera_calibration_and_3d_reconstruction.html} to understand.
		}
		\logentry{4}{19}{2016}{Made adjustments to python for image acquisition scripts (from blogs mentioned on \formatdate{11}{4}{2016}.) \textbf{NOTE:} Consider creating rig with glue to keep stereo camera placement / direction constant.}
		\logentry{4}{19}{2016}{\textbf{UPDATE:} Error with \texttt{calibrate\_cameras} python code causing linux machine to crash. If can't be resolved switch over to MacBook. \textbf{NOTE:} Package should be setup by calling \texttt{\$ python setup.py install}}
		\logentry{4}{19}{2016}{\textbf{UPDATE:} Crash due to recursive shell call and was fixed. OpenCV not detecting all chessboard corners. Will try a new board.}
		\logentry{4}{20}{2016}{Did small amount of work on \textbf{Change of Reference} section in the paper. Added a section to the intro containing a map of commonly used symbols and notation}
		\logentry{4}{29}{2016}{Read following sections of [Chen93]~\cite{Chen93}:
			\begin{itemize}
				\item Abstract
				\item Introduction
				\item Visibility Morphing
			\end{itemize}
			\textbf{Summary:} Explicit Geometry is ignored (i.e. surface mesh and 3d-points). Geometry is kept in 2-d. 
			Whereas Image Morphing interpolates between \textit{pixel intensity values in fixed locations} the method in this article interpolates between 
			\textit{pixel locations with (relatively) fixed intensity values}.
			\textbf{Question:} Sections read mention that pixel positions are stored in 3d (3-tuple) data structure. I'm not sure I understand this correctly, since 
			\begin{enumerate}
				\item This would effectively make this structure a point cloud (but no mention of it in the paper).
				\item There is no mention of special "depth-based" hardware or cameras (Far as I know this is upposed to be a regular image).
			\end{enumerate}
		}
		\logentry{4}{30}{2016}{Checked understanding of \textit{epipolar constraint} through reading of [Hartley2004]~\cite{Hartley2004} and its derivation of 
			\begin{equation*}
				\begin{split}
					{\mathbf{'x}^T}\cdot{\mathbf{E}}\cdot\mathbf{x} &= 
					{\mathbf{'x}^T}\cdot{\lbrack\mathbf{t}\rbrack}_{\times}\cdot{\mathbf{R}}\cdot\mathbf{x} \\
					&= {\mathbf{'x}^T}\cdot{'l}
				\end{split}
			\end{equation*}
			and creation of MatLab code verifying this.\newline
			\par I may have been mistaken about relation of \textbf{Fundamental Matrix} and \textbf{Essential Matrix}. \newline
			\par My current understanding is the \textit{Fundamental Matrix} describes point/epipolar line correspondance for images under \textbf{scale invariant} conditions (i.e. point correspondance and Fundamental matrix does not change when one image (or both images) are scaled (uniformly or omni-directionally). \newline
			\par \textit{Essential Matrix} describes point/epipolar line correspondance for images under \textbf{normalized} conditions (i.e. unit-length is set equal to focal-length, and projection center is set at $(0,0,1)$.\newline 
		}
		\logentry{5}{2}{2016}{
			Additional wording to Stereo-vision section. I am unsure of best order to present ideas related to \textit{multi-view} geometry.
		}
	\end{longtable}




	\newpage

	%% Code below should be used for citations 

%	Longuet-Higgins' Fundamental Matix ~\cite{Longuet-Higgins}
	\bibliography{citations}{}
	\bibliographystyle{unsrt}

\end{document}
