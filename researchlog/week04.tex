%%%%%%%%%%%%%%%%%%%%%%%%%%%%%%%%%%%%%%
%% begin week04.tex
%%%%%%%%%%%%%%%%%%%%%%%%%%%%%%%%%%%%%%
	\begin{longtable}{l p{12cm} }

		\logentry{6}{5}{2016}{%
Almost done with MatLab triangle interpolation program. Hoping to have something to show Kamangar in the next few days.\newline
\par Was reading up on image-segmentation as a way to improve feature detection through masking. Came accross references to \textbf{spectral clustering} which I still don't understand after data mining class. Was reading tutorial at \url{http://classes.engr.oregonstate.edu/eecs/spring2012/cs534/notes/Spectral.pdf} for starters.
		}
		\logentry{6}{8}{2016}{%
Finalized most recent changes to MatLab program. It performs interpolation (between \textit{source} and \textit{destination} images of triangular patches defined by Delaunay triangularization of point correspondances from stereo images (See \texttt{Wood\_Kamangar/StatusReports/StatusReport\_00/Images}). Delaunay triangularization is performed on the source image only then extended to the corresponding points in the destination image so the arrangement of Delaunay triangles remains the same between images.\newline
\par Summary of results is as follows:
\begin{itemize}
\item Triangles confined to one disparity region (See statue head in \texttt{image\_source.png}, \texttt{image\_destination.png}, and \texttt{truedisp.row3.col3.pgm}) show few artifacts and minimal blurring.
\item Triangles crossing disparity regions or containing pixels occluded in the source or destination images (see camcorder tripod and lamp stand) have visibly more artifacts.\newline
\end{itemize}
\par Started reading first page (\textit{Abstract} and \textit{Introduction} sections) of [Sharstein2002]~\cite{Scharstein2002}.
		}
		\logentry{6}{9}{2016}{%
Continuing to read [Scharstein2002]~\cite{Scharstein2002}.\newline
\par
\SUMMARY{
Disparity can be defined by two ideas:
\begin{itemize}
\item \textit{Human Vision} : Difference in location of features in the left and right eye.
\item \textit{Computer Vision} : Inverse depth. Can be treated as a 3-dimensional projective transformation (collineation or homographyv)of 3-d space (X,Y,Z).
\newline
\end{itemize}
\par Define fllowing terms:
\begin{itemize} 
\item \textbf{Disparity Map}: $d(x,y)$
\item \textbf{Disparity Space}: $(x,y,d)$
\item \textbf{Correspondance}: Pixel $(x,y)$ in reference image $r$ and corresponding pixel $(x',y')$ in matching image $m$ given by $x' = x + s d(x,y)$ and$y' = y$ (assuming horizontal displacement \textit{only}), where $s = \pm 1$ is chose do $d$ is always positive. 
\item \textbf{Disparity Space Image}: Any function or image defined over continous or dispartiy space.
\newline
\end{itemize}
}
		}
		\logentry{6}{11}{2016}{%
Continuing to read [Scharstein2002]~\cite{Scharstein2002}:\newline
\par
\SUMMARY{
Algorithms can be ordered in 4 common subsets:
\begin{enumerate}
\item Matching cost computation;
\item Cost (support) aggregation;
\item Disparity computation / optimization;
\item Disparity refinement;\newline 
\end{enumerate}
\par Two main types of agorithms:
\begin{itemize}
\item \textbf{Local}: Including \textit{Squared Intensity Differences} and \textit{Absolute intensity differences}.
\item \textbf{Global} Includeing \textit{Energy minimizatio}.\newline
\end{itemize}
}
\par Continuing to read up on \textit{Spectral Clustering} and \textit{Laplacian embedding} for uses in image segmentation.
		}
	\end{longtable}
