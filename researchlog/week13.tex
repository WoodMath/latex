%%%%%%%%%%%%%%%%%%%%%%%%%%%%%%%%%%%%%%
%% begin week13.tex
%%%%%%%%%%%%%%%%%%%%%%%%%%%%%%%%%%%%%%
	\begin{longtable}{l p{12cm} }
	\logentry{8}{7}{2016}{%
Worked on Python program OpenGL aspects for implmenting [Fusiello1999]~\cite{Fusiello1999} in Python.\newline
	}
	\logentry{8}{8}{2016}{%
Started reading [Hong2004]~\cite{Hong2004}. It was a little over my head. After looking for a tutorial online I found \url{https://www.inf.ethz.ch/personal/ladickyl/CVPR_Tutorial2015.htm}, which is based on [Boykov2001]~\cite{Boykov2001}. I added it to my reading list.\newline

\par Revamped working of Python demo program, and worked on additional coding.
	}
	\logentry{8}{9}{2016}{%
I spent most of the day working some more on \textit{Demo program}. Spent a little time reading [Hartley2004]~\cite{Hartley2004}.\newline

\par \SUMMARY{%
Relating to \textit{Projective Geometry} discussed on \formatdate{29}{6}{2016}, \textit{Points at infinity} are all points $\mathbf{P}_\infty=[x_1,x_2,0]^\intercal$ such that $x_3=0$. All such points lie on a single line $\mathbf{l}_\infty=[0,0,1]^\intercal$ referred to as a \textit{line at infinity}. A \textit{point at infinity} and \textit{line at infinity} can be mapped to a \textit{finite point} and \textit{finite plane} via a \textit{projective transformation} but lie fixed at \textit{infinity} under an \textit{affine transformation}.\newline
}
	}
	\logentry{8}{11}{2016}{%
\UPDATE{%
Started coding process for \textit{spectral clustering} detailed on \formatdate{
5}{8}{2016}. Completed items on \textbf{1. Downsample original image and perform spectral clustering}, \textbf{3. Partition original size image}. I still need to code \textbf{5. Join segmented sub areas}. Majority of \textbf{2. Perform spectral clustering down sampled image} and \textbf{4. Perform spectral clustering on sub-area images} items had previously been coded before issues with memory limitiations had been discovered.\newline

\par I put in an additional help-ticket to \texttt{MatLab} support regarding issues logging into \texttt{MathWorks} cloud.  

}
	}
	
%	\hline
	\end{longtable}

