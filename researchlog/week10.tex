%%%%%%%%%%%%%%%%%%%%%%%%%%%%%%%%%%%%%%
%% begin week10.tex
%%%%%%%%%%%%%%%%%%%%%%%%%%%%%%%%%%%%%%
	\begin{longtable}{l p{12cm} }
	\logentry{7}{17}{2016}{%
Spent a couple of hours working on \textit{demonstration} code in OpenGL and OpenCV.
	}
	\logentry{7}{18}{2016}{%
Spending day working on thesis document. Sections worked on include:\newline
\par
\begin{itemize}
\item Intrinsic Calibration Matrix
\item Fundamental Matrix
\end{itemize}
	}
	\logentry{7}{19}{2016}{%
Continuing to add material to thesis document, including:\newline
\par
\begin{itemize}
\item Extrinsi Calibration Matrix
\item Fundamental Matrix\newline
\end{itemize}
\par Going back to reread first parts of Chapter 6 from [Hartley2004]~\cite{Hartley2004}, as I need clarification on some aspects of the \textit{calibration matrix}. Namely, I \textit{still} do not understand how $\mathbf{X}(\lambda)=\mathbf{P}^+\mathbf{x}+\lambda\mathbf{C}$ represents the equation of a ray passing through \textit{optical center} $\mathbf{C}$ in \textit{world space}, with \textit{projection matrix} $\mathbf{P}$.
	}
	\logentry{7}{20}{2016}{%
Added material on \textit{fundamental matrix calculation from data} to thesis document. Reading additional material from [Hartley2004]~\cite{Hartley2004} on \textit{fundamenta matrix theoretical calculation}.
	} 
	\logentry{7}{21}{2016}{%
Continuing to read [Martin2008]~\cite{Martin2008}. See questions below.\newline

\par \Kamangar{%
I don't understand the difference between \textit{forward mapping} and \textit{backward mapping}.
}\newline

\par I'm a bit confused about most of the material being discussed in [Martin2008]~\cite{Martin2008}. Will read [Karathanasis1996]~\cite{Karathanasis1996} for background on \textit{disparity estimation using dynamic programming}.\newline

\par \UPDATE{%
My question on \formatdate{13}{7}{2016} may have been worded wrong: The \textit{dynamic programming} is used for estimating \textit{disparity}, which in turn is used for \textit{point correspondance}. The \textit{dynamic programming} is not used DIRECTLY, in calcuating \textit{point correspondance}.}\newline

\par Orignal question still holds though:\newline

\par \Kamangar{% 
I understand \textit{ALL} of the following to be \textit{TRUE}, which one needs to be \textit{FALSE} (or my understanding revised):
\begin{itemize}
\item \textit{Point correspondence} is needed to compute \textit{rectifying homographies}.
\item \textit{Rectifying homography} is needed to compute \textit{disparities}.
\item \textit{Disparity} is needed to compute \textit{point correspondence}.
\end{itemize} 
}

	}
	\logentry{7}{22}{2016}{%
Started reading [Karathanasis1996]~\cite{Karathanasis1996}, no new information from first few sections.
	}
	\end{longtable}
