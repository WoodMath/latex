%%%%%%%%%%%%%%%%%%%%%%%%%%%%%%%%%%%%%%
%% begin week01.tex
%%%%%%%%%%%%%%%%%%%%%%%%%%%%%%%%%%%%%%
	\begin{longtable}{l p{12cm} }
%\newpage\\\hline %% e-mail 20160523
		\logentry{5}{18}{2016}{Reviewed [Chen1993]~\cite{Chen1993} Section 2. Consider reviewing follow relevant articles:\newline
\par
			\begin{itemize}
				\item Disparity [Gosh89]
				\item Optical Flow [Nage86]
				\item Look-up tables [Wolb89]
				\item 3d scenes [Pogg91]\newline
			\end{itemize}
\par Working on MatLab code to pick correspondig points in stereo-images, and calculate pixel offset vectors.
		}
		\logentry{5}{19}{2016}{Read Section 2.3 of [Chen1993]~\cite{Chen1993}. View interpolation is limited by:\newline
\par
			\begin{itemize}
				\item \textbf{Penumbra}: pixels visible in one source image \textit{but not both}
				\item \textbf{Umbra}, pixels visible in neither source image, and \textit{invisible} in destination image.
				\item \textbf{Holes}, pixels visible in neither source image, but \textit{visible} in destination image.\newline
			\end{itemize}
\par Calculatred formula for \textit{pre-displaced} quad-pixel calculation using a bi-linear interpolation as:
\begin{equation*}
\mathbf{P}(u,v) = 
\mathbf{P}(0,0)\cdot (1-u)\cdot (1-v)+\mathbf{P}(1,0)\cdot u \cdot (1-v)+
\mathbf{P}(0,1)\cdot (1-u)\cdot v +\mathbf{P}(1,1)\cdot u \cdot v
\end{equation*}
		}
		\logentry{5}{20}{2016}{
Derived formula for 
\textit{uv} calculation using 
\textit{geometry matrix}, \textit{blending matrix} and
\textit{basis vectors} of 
$\mathbf{u}=[u\ 1]^{T}$ and 
$\mathbf{v}=[v\ 1]^{T}$


\begin{equation*}\begin{split}
x_{uv} &= 
\begin{bmatrix}u & 1\end{bmatrix}
\begin{bmatrix}-1 & 1\\ 1 & 0\\\end{bmatrix}
\begin{bmatrix}x_{00} & x_{01} \\ x_{10} & x_{11}\\\end{bmatrix}
\begin{bmatrix}-1 & 1\\ 1 & 0\\\end{bmatrix}
\begin{bmatrix}v \\ 1\\\end{bmatrix}\\
y_{uv} &= 
\begin{bmatrix}u & 1\end{bmatrix}
\begin{bmatrix}-1 & 1\\ 1 & 0\\\end{bmatrix}
\begin{bmatrix}y_{00} & y_{01} \\ y_{10} & y_{11}\\\end{bmatrix}
\begin{bmatrix}-1 & 1\\ 1 & 0\\\end{bmatrix}
\begin{bmatrix}v \\ 1\\\end{bmatrix}\\
\end{split}\end{equation*}

			\par\Kamangar{Is there a way given $x$ and $y$ to solve for $u$ and $v$?}
		}
	\end{longtable}
