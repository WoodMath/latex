\documentclass[a4paper,10pt]{article}
%\documentclass[fleqn]{article}
%\documentclass[a4paper,10pt]{scrartcl}
%\setlength\parindent{0pt}
\usepackage[utf8]{inputenc}
\usepackage{amssymb}
\usepackage{amsmath}
\usepackage{amsfonts}
\usepackage[margin=0.5in]{geometry}
\usepackage{hyperref}

\newcommand{\sNHu}{\ensuremath{\textbf{u}}}
\newcommand{\sHu}{\ensuremath{\textbf{\~{u}}}}
\newcommand{\sNHx}{\ensuremath{\textbf{x}}}
\newcommand{\sHx}{\ensuremath{\textbf{\~{x}}}}
\newcommand{\sNHcx}{\ensuremath{\textbf{^CX}}}
\newcommand{\sHcx}{\ensuremath{\textbf{^C\~{X}}}}
\newcommand{\sNHK}{\ensuremath{\textbf{K}}}
\newcommand{\sHK}{\ensuremath{\textbf{\~{K}}}}
\newcommand{\sNHPi}{\ensuremath{\textbf{\Pi_0}}}
\newcommand{\sHPi}{\ensuremath{\textbf{\~{\Pi_0}}}}
\newcommand{\vNHuv}{\ensuremath{\left[\begin{array}{c}u\\v\\\end{array}\right]}}
\newcommand{\vHuv}{\ensuremath{\left[\begin{array}{c}u\\v\\1\\\end{array}\right]}}
\newcommand{\vNHxy}{\ensuremath{\left[\begin{array}{c}x\\y\\\end{array}\right]}}
\newcommand{\vHxy}{\ensuremath{\left[\begin{array}{c}x\\y\\1\\\end{array}\right]}}
\newcommand{\vNHxyz}{\ensuremath{\left[\begin{array}{c}x\\y\\z\\\end{array}\right]}}
\newcommand{\vHxyz}{\ensuremath{\left[\begin{array}{c}x\\y\\z\\1\\\end{array}\right]}}
\newcommand{\mHtoNHtwoD}{\ensuremath{\left[\begin{array}{ccc}1 & 0 & 0\\0 & 1 & 0\\\end{array}\right]}}
\newcommand{\mHtoNHthrD}{\ensuremath{\left[\begin{array}{cccc}1 & 0 & 0 & 0\\0 & 1 & 0 & 0\\0 & 0 & 1 & 0\\\end{array}\right]}}
\newcommand{\mIzero}{\ensuremath{\left[\begin{array}{cc}\textbf{I} & \textbf{0}\\\end{array}\right]}}
\newcommand{\mIone}{\ensuremath{\left[\begin{array}{c}\textbf{I} \\ \textbf{1}\\\end{array}\right]}}
\newcommand{\mNHkuv}{\ensuremath{\left[\begin{array}{ccc}k_u & 0 & u_0\\0 & k_v & v_0\\\end{array}\right]}}
\newcommand{\mHkuv}{\ensuremath{\left[\begin{array}{ccc}k_u & 0 & u_0\\0 & k_v & v_0\\ 0 & 0 & 1\\\end{array}\right]}}
\newcommand{\mNHf}{\ensuremath{\left[\begin{array}{ccc}f & 0 & 0\\0 & f & 0\\0 & 0 & 1\\\end{array}\right]}}
\newcommand{\vNHcx}{\ensuremath{\left[\begin{array}{c}^CX\\^CY\\^CZ\\\end{array}\right]}}
\newcommand{\vHcx}{\ensuremath{\left[\begin{array}{c}^CX\\^CY\\^CZ\\1\\\end{array}\right]}}
\newcommand{\mNHfkuv}{\ensuremath{\left[\begin{array}{ccc}fk_u & 0 & u_0\\0 & fk_v & v_0\\\end{array}\right]}}
\newcommand{\mHfkuv}{\ensuremath{\left[\begin{array}{ccc}fk_u & 0 & u_0\\0 & fk_v & v_0\\ 0 & 0 & 1\\\end{array}\right]}}
%\newcommand{\mRTzeroone}{\ensuremath{\left[\begin{array}{cc}\mathbf{R}&\mathbf{t}\\\mathbf{0}&\mathbf{1}\\\end{array}\right]}}
\newcommand{\mRTzeroone}{\ensuremath{\left[\begin{array}{c|c}\mathbf{R}&\mathbf{t}\\\hline\mathbf{0}&\mathbf{1}\\\end{array}\right]}}
\newcommand{\sphx}[1]{\ensuremath{{^{#1}}\mathbf{\tilde{x}}}}
\newcommand{\spphx}[2]{\ensuremath{{^{#1}_{#2}}\mathbf{\tilde{x}}}}
\newcommand{\sppphx}[3]{\ensuremath{{^{#1}_{#2}}\mathbf{\tilde{{#3}}}}}
\newcommand{\spnhx}[1]{\ensuremath{{^{#1}}\mathbf{{x}}}}
\newcommand{\sppnhx}[2]{\ensuremath{{^{#1}_{#2}}\mathbf{{x}}}}
\newcommand{\spppnhx}[3]{\ensuremath{{^{#1}_{#2}}\mathbf{{{#3}}}}}

%\newcommand{\mRTzeroone}{\ensuremath{\left[\begin{array}{cc}\mathbf{R}&\mathbf{t}\\{\vec{0}}^{T}&1\\\end{array}\right]}}
%\newcommand{\so}[1]{{#1}_{Surge\;Occurs}}
%\newcommand{\nsb}[1]{\overline{{#1}_{Surge\;Blocked}}}
%\newcommand{\nt}[1]{N\left({#1}\right)}
%$\newcommand{\WM_xy}{\mathrm{\left[\begin{array}{c}
%x\\
%y\\
%\end{array}\right]}}$

\title{Geometric Formulas for Computer Vision and Computer Graphics}
\author{JeffGWood@mavs.uta.edu}
\date{Fall 2014}

\pdfinfo{%
  /Title    (Geometric Formulas for Computer Vision and Computer Graphics)
  /Author   (Jeff Wood)
  /Creator  ()
  /Producer ()
  /Subject  ()
  /Keywords ()
}

\begin{document}
%\newline\noindent\newline\noindent
\Huge\textbf{Geometric Formulas for \newline Computer Vision and Computer Graphics}\newline
\newline\noindent\newline\noindent
\large Written by Jeff Wood in LaTeX environment\newline
\large Updated 2014/10/29 \newline
\normalsize
\newline\noindent\newline\noindent
\textbf{Sources include:}\newline
\url{http://ranger.uta.edu/~gianluca/teaching/CSE4392-5369_F14/3_CSE4392-5369_IntroVision_Mariottini.pdf}\newline
\url{http://ranger.uta.edu/~gianluca/teaching/CSE4392-5369_F14/6_CSE4392-5369_RigidBodyTransform_Mariottini.pdf}\newline
\url{http://ranger.uta.edu/~gianluca/teaching/CSE4392-5369_F14/7_CSE4392-5369_CameraCalibrationResectioning_Mariottini.pdf}\newline
\url{http://ranger.uta.edu/~gianluca/teaching/CSE4392-5369_F14/10_CSE4392-5369_EpipolarGeometry_Mariottini.pdf}
\newline\noindent\newline\noindent
%\textbf{Homogeneous ($\tilde{x}$) to Non-Homogeneous ($x$) - 2-d case:}
%\textbf{Homogeneous (\~{x}) to Non-Homogeneous (x) - 2-d case:}
\textbf{Homogeneous (\~{x}) to Non-Homogeneous (x) - 2-d case:}
\begin{equation}
\begin{split}
\sNHx = \vNHxy=\mHtoNHtwoD \vHxy=\mIzero \sHx
%\Longrightarrow
\end{split}
\end{equation}
\newline\noindent\newline\noindent
%\textbf{Homogeneous ($\tixylde{x}$) to Non-Homogeneous ($x$) - 3-d case:}
%\textbf{Homogeneous (\~{x}) to Non-Homogeneous (x) - 3-d case:}
\textbf{Homogeneous (\~{x}) to Non-Homogeneous (x) - 3-d case:}
\begin{equation}
\begin{split}
\sNHx =\vNHxyz=\mHtoNHthrD\vHxyz=\mIzero \sHx
%\Longrightarrow
\end{split}
\end{equation}
\newline
\textbf{Camera Resolution:}
\begin{equation}
\begin{split}
\sNHu = \vNHuv &= \mNHkuv\vHxy = \mNHkuv\sHx\\
& \Downarrow 
\\
\sHu = \vHuv &= \mHkuv\vHxy = \mHkuv\sHx\\
\end{split}
\end{equation}
\newline\noindent\newline\noindent
\textbf{Camera Focal Length:}
\newline\noindent\newline\noindent
Using \emph{Similarity of Triangles} we get $\dfrac{x}{f} = \dfrac{^CX}{^CZ}$: 
\begin{equation}
\begin{split}
s\sHx = s\vHxy &= \mNHf\vNHcx \\
& \Downarrow \\
\sHx = \vHxy &= \mNHf\vNHcx\lambda \\
\end{split}
\end{equation}
Where $s\;=\;^CZ\;=\;\dfrac{1}{\lambda}$.
\newpage\noindent
\textbf{Camera Calibration ($\sNHK$):}
\newline\noindent\newline\noindent
\begin{equation}
\begin{split}
\sHu &= \mHkuv\sHx = \mHkuv\left(\vHxy\right)=\mHkuv\left(\mNHf\vNHcx\lambda\right)\\
&=\left(\mHkuv\mNHf\right)\vNHcx\lambda=\mHfkuv\vNHcx\lambda=\sNHK\vNHcx\lambda\\\
\end{split}
\end{equation}
\newline\noindent\newline\noindent
\textbf{Ideal Projection Matrix ($\Pi_0$):}
\newline\noindent\newline\noindent
\begin{equation}
\begin{split}
\sHu &= \mHkuv\sHx = \mHkuv\left(\vHxy\right)=\mHkuv\left(\mNHf\vNHcx\lambda\right)\\
&=\left(\mHkuv\mNHf\right)\vNHcx\lambda=\mHfkuv\vNHcx\lambda=\sNHK\vNHcx\lambda\\\
\end{split}
\end{equation}
\noindent\textbf{Geometric Tranformations:}
\newline\noindent\newline\noindent
%Using \emph{Similarity of Triangles} we get $\dfrac{x}{f} = \dfrac{^CX}{^CZ}$: 
\mRTzeroone
\newline\noindent\newline\noindent
\textbf{Cross Product (Skew-Symmetric Form):}
\newline\noindent\newline\noindent
\begin{equation*}
\begin{split}
\mathbf{a}\times\mathbf{b} &= 
\left[\begin{array}{c}a_{1}\\a_{2}\\a_{3}\\\end{array}\right]
\times
\left[\begin{array}{c}b_{1}\\b_{2}\\b_{3}\\\end{array}\right]
\\       
{[\mathbf{a}]}_{\times}{\mathbf{b}}
&=
\left[\begin{array}{rrr}
0 & -a_{3} & a_{2}\\
a_{3} & 0 & -a_{1}\\
-a_{2} & a_{1} & 0\\
      \end{array}\right]
\left[\begin{array}{c}b_{1}\\b_{2}\\b_{3}\\\end{array}\right]      
\end{split}
\end{equation*}
\newline\noindent
\textbf{Change of Frame (of Reference):}
\newline\noindent
Rearranging
\begin{equation*}
\begin{split}
\spnhx{A} &= \spppnhx{A}{B}{R}\spnhx{B} + \spppnhx{A}{B}{t}\\
&\Downarrow\\
\spnhx{A} - \spppnhx{A}{B}{t} &= \spppnhx{A}{B}{R}\spnhx{B}\\
&\Downarrow\\
\spnhx{B} &= {\spppnhx{A}{B}{R}}^{-1}\left(\spnhx{A} - \spppnhx{A}{B}{t}\right)\\
&= {\spppnhx{A}{B}{R}}^{T}\left(\spnhx{A} - \spppnhx{A}{B}{t}\right)\\
&= {\spppnhx{A}{B}{R}}^{T}\spnhx{A} - {\spppnhx{A}{B}{R}}^{T}\spppnhx{A}{B}{t}\\
\end{split}
\end{equation*}
\noindent Implies %(for ${^{B}\mathbf{x}} = {^{B}_{A}\mathbf{R}}{^{A}\mathbf{x}} + {^{B}_{A}\mathbf{t}}$)
\begin{equation*}
\begin{split}
\spppnhx{B}{A}{R}={\spppnhx{A}{B}{R}}^{T}
\indent\indent
\text{and}
\indent\indent
\spppnhx{B}{A}{t}= -{\spppnhx{A}{B}{R}}^{T}\spppnhx{A}{B}{t}\\
\end{split} 
\end{equation*}
\noindent Where
\begin{equation*}
\begin{split}
\spnhx{B} &= \spppnhx{B}{A}{R}\spnhx{A} + \spppnhx{B}{A}{t}\\
\end{split} 
\end{equation*}
\newline\noindent\newline\noindent
\newpage\noindent
\textbf{Essential Matrix (Theoretical Calculation):}
\newline\noindent\newline\noindent
Relationship between $^{C}\mathbf{x}$ and $^{C'}\mathbf{x}$:
\newline\noindent\newline\noindent
\begin{equation*}
\begin{split}
{^{C'}\mathbf{x}} = {\spppnhx{C'}{C}{R}}{^{C}\mathbf{x}}+{\spppnhx{C'}{C}{t}}
\end{split}
\end{equation*}
\newline\noindent\newline\noindent
Taking the \emph{cross-product} with $[\spppnhx{C'}{C}{t}]_{\times}$:
\newline\noindent\newline\noindent
\begin{equation*}
\begin{split}
{[\spppnhx{C'}{C}{t}]_{\times}}{^{C'}\mathbf{x}} &= 
{[\spppnhx{C'}{C}{t}]_{\times}}{\spppnhx{C'}{C}{R}}{^{C}\mathbf{x}}+
{[\spppnhx{C'}{C}{t}]_{\times}}{\spppnhx{C'}{C}{t}}\\
&={[\spppnhx{C'}{C}{t}]_{\times}}{\spppnhx{C'}{C}{R}}{^{C}\mathbf{x}}+\mathbf{0}\\
&={[\spppnhx{C'}{C}{t}]_{\times}}{\spppnhx{C'}{C}{R}}{^{C}\mathbf{x}}\\
\end{split}
\end{equation*}
\newline\noindent\newline\noindent
Multiplying (on the left) by $^{C'}\mathbf{x}^{T}$:
\newline\noindent\newline\noindent
\begin{equation*}
\begin{split}
{^{C'}\mathbf{x}^{T}}{[\spppnhx{C'}{C}{t}]_{\times}}{^{C'}\mathbf{x}} &=
{^{C'}\mathbf{x}^{T}}{[\spppnhx{C'}{C}{t}]_{\times}}{\spppnhx{C'}{C}{R}}{^{C}\mathbf{x}}\\
\end{split}
\end{equation*}
Since $^{C'}\mathbf{x}^{T}$ is \emph{orthogonal} to ${[\spppnhx{C'}{C}{t}]_{\times}}{^{C'}\mathbf{x}}$ the above is equal to
\begin{equation*}
\begin{split}
{^{C'}\mathbf{x}^{T}}{[\spppnhx{C'}{C}{t}]_{\times}}{^{C'}\mathbf{x}} &=
{^{C'}\mathbf{x}^{T}}{[\spppnhx{C'}{C}{t}]_{\times}}{\spppnhx{C'}{C}{R}}{^{C}\mathbf{x}}\\
{^{C'}\mathbf{x}^{T}}\left({[\spppnhx{C'}{C}{t}]_{\times}}{^{C'}\mathbf{x}}\right) &=
{^{C'}\mathbf{x}^{T}}\left({[\spppnhx{C'}{C}{t}]_{\times}}{\spppnhx{C'}{C}{R}}\right){^{C}\mathbf{x}}\\
\mathbf{0}&={^{C'}\mathbf{x}^{T}}{^{C'}_{C}\mathbf{E}}{^{C}\mathbf{x}}\\
\end{split}
\end{equation*}
\newline\noindent\newline\noindent
\textbf{Essential Matrix (Practical Calculation):}
\newline\noindent
\end{document}